\documentclass[11pt,a4paper]{article}

% =========================
% Packages
% =========================
\usepackage[utf8]{inputenc}
\usepackage[T1]{fontenc}
\usepackage{lmodern}

\usepackage{amsmath, amssymb}
\usepackage{graphicx}
\usepackage{float}
\usepackage{booktabs}
\usepackage{pdfpages}

\usepackage{hyperref}
\hypersetup{
    colorlinks=true,
    linkcolor=blue,
    urlcolor=blue,
    citecolor=blue
}

\usepackage[margin=1in]{geometry}
\usepackage{setspace}
\onehalfspacing

% =========================
% Title Information
% =========================
\title{Example LaTeX Document}
\author{Manuel Santos}
\date{\today}

% =========================
% Document
% =========================
\begin{document}

\maketitle

\begin{abstract}
This document serves as an example \LaTeX{} \texttt{main.tex} file.
It demonstrates a common structure used in academic and technical writing.
\end{abstract}

\section{Introduction}

This is the introduction section.  
\LaTeX{} is widely used for scientific documents due to its excellent
typesetting quality and mathematical support.

\section{Mathematics Example}

Inline math example: \( E = mc^2 \).

Displayed equation:
\begin{equation}
    \int_{0}^{\infty} e^{-x^2} \, dx = \frac{\sqrt{\pi}}{2}
\end{equation}

\section{Figures}

Figure example:

\begin{figure}[H]
    \centering
    \includegraphics[width=0.6\textwidth]{Images/LaTeX.jpg}
    \caption{Example figure}
    \label{fig:example}
\end{figure}

\section{Tables}

\begin{table}[H]
    \centering
    \begin{tabular}{l c r}
        \toprule
        Item & Quantity & Price (€) \\
        \midrule
        Apples  & 3 & 2.50 \\
        Oranges & 5 & 3.20 \\
        Bananas & 2 & 1.10 \\
        \bottomrule
    \end{tabular}
    \caption{Example table}
    \label{tab:example}
\end{table}

\section{Conclusion}

This example demonstrates:
\begin{itemize}
    \item Document structure
    \item Math typesetting
    \item Figures and tables
\end{itemize}

And finally, a citation to an important work \cite{very_important}.

\section*{Acknowledgements}

Optional acknowledgements go here.

% =========================
% Bibliography (optional)
% =========================
\section*{Bibliography}
\bibliographystyle{plain}
\bibliography{references}

\section*{Appendix}
As an appendix, I leave you with a dummy PDF inclusion.
\includepdf[pages=-]{PDF_to_include/dummy.pdf}

\end{document}
